\documentclass{VUMIFPSkursinis}
\usepackage{algorithmicx}
\usepackage{algorithm}
\usepackage{algpseudocode}
\usepackage{amsfonts}
\usepackage{amsmath}
\usepackage{bm}
\usepackage{caption}
\usepackage{color}
\usepackage{float}
\usepackage{graphicx}
\usepackage{listings}
\usepackage{subfig}
\usepackage{wrapfig}

\usepackage{enumitem}
%PAKEISTA, tarpai tarp sąrašo elementų
\setitemize{noitemsep,topsep=0pt,parsep=0pt,partopsep=0pt}
\setenumerate{noitemsep,topsep=0pt,parsep=0pt,partopsep=0pt}

% Titulinio aprašas
\university{Vilniaus universitetas}
\faculty{Matematikos ir informatikos fakultetas}
\department{Programų sistemų katedra}
\papertype{Programų sistemų inžinerijos I laboratorinis darbas}
\title{Socialinis Vilniaus universiteto tinklalapis}
\titleineng{SocialVU}
\status{2 kurso 4 grupės studentai}
\author{Andrejus Voitovas}
\secondauthor{Eglė Puodžiūnaitė}
\thirdauthor{Kasparas Kralikas}
\fourthauthor{Ieva Vizgirdaitė} % Pridėti antrą autorių
\supervisor{asist. dr. Vytautas Valaitis}
\date{Vilnius – \the\year}

% Nustatymai
% \setmainfont{Palemonas}   % Pakeisti teksto šriftą į Palemonas (turi būti įdiegtas sistemoje)
\bibliography{bibliografija}

\begin{document}
	
% PAKEISTA	
\maketitle
\cleardoublepage\pagenumbering{arabic}
\setcounter{page}{2}

%TURINYS
\tableofcontents

\sectionnonum{Įvadas}
Tikslas - sukurti socialinio tinklalalapio prototipą, kurį įgyvendinus būtų palengvinta universiteto bendruomenės komunikaciją.

Temos aktualumas:
Šiuo metu studentams dėstytojų skelbiama informacija yra išbarstyta internete, kurią surasti užima galybes laiko. Yra atskiras universiteto naujienų puslapis, kiekvienas dėstytojas turi savo asmeninį tinklalapį, atskiras elektroninis paštas. Tiek dėstytojui pasiekti studentus, tiek studentui dėstytoją yra komplikuota ir nepatogu.

Siekiami rezultatai:

\sectionnonum{VERSLO PROCESO APRAŠAS}
\section{IŠORINĖ VERSLO PROCESO ANALIZĖ}
Išorinė analizės metu išsiaiškinami ištekliai
reikalingi pradėti verslą, išryškinamos kuriamo verslo grėsmės ir
problemos, kuriami tinklalapio teikiami rezultatai ir išanalizuojama rinkoje esanti situacija. Nustačius rinkoje esančius ir potencialius konkurentus išanalizuojamos
neišnaudotos galimybės bei įvertinama situacija.
\begin{center}
\subsection{Įeigos}
\end{center}
\begin{enumerate}
	\item Studentai
	\item Dėstytojai
	\item Administratorius
	\item Duomenų bazė
	\item Serveris
\end{enumerate}
\begin{table}[H]\footnotesize
	\centering
	\caption{Įeigos}
	{\begin{tabular}{|c|c|c|c|} \hline
			Vertinimo kriterijus & Mertinimo matas & Siekiama vertė & Kritinė vertė \\
			\hline
			Studentai & Kiekis studentų, naudojančių socialVU tinklalapį & 21281 & 10000 \\
			\hline
			Dėstytojai & Kiekis dėstytojų, naudojančių socialVU tinklalapį & 2890 & 1500 \\
			\hline
			Administratorius & Pašalintų naudotojų skaičius per
			mėnesį & 0 & 30 \\
			\hline
			Duomenų bazė & Talpa GB & 50 & 10 \\
			\hline
			Serveris & Vidutinis užklausos apdorojimo laikas sekundėmis & 1 & 5 \\
			\hline
	\end{tabular}}
\end{table}
\begin{center}
\subsection{Išeigos}
\end{center}
\begin{enumerate}
	\item Studentai, radę visą reikiamą informaciją
	\item Dėstytojai, pateikę visą norimą informaciją
\end{enumerate}
\begin{table}[H]\footnotesize
	\centering
	\caption{Išeigos}
	{\begin{tabular}{|c|c|c|c|} \hline
			Vertinimo kriterijus & Mertinimo matas & Siekiama vertė & Kritinė vertė \\
			\hline
			Studentai, radę visą reikiamą informaciją & Kiekis per mėnesį \% & 100 & 50 \\
			\hline
			Dėstytojai, pateikę visą norimą informaciją & Kiekis per mėnesį \% & 100 & 50 \\
			\hline
	\end{tabular}}
\end{table}
\begin{center}
	\subsection{Įvaizdis}
\end{center}
\begin{enumerate}
	\item Dėstytojų bei studentų atsiliepimai
	\item Sistemos žinomumas
\end{enumerate}
\begin{table}[H]\footnotesize
	\centering
	\caption{Įvaizdis}
	{\begin{tabular}{|c|c|c|c|} \hline
			Vertinimo kriterijus & Mertinimo matas & Siekiama vertė & Kritinė vertė \\
			\hline
			Dėstytojų bei studentų atsiliepimai & Teigiamų atsiliepimų \% nuo visų atsiliepimų & 90 & 50 \\
			\hline
			Sistemos žinomumas & Vilniaus universiteto studentų \% žinančių šį tinklalapį & 100 & 50 \\
			\hline
	\end{tabular}}
\end{table}
\begin{center}
	\subsection{Reguliavimas}
\end{center}
\begin{enumerate}
	\item Asmens duomenų teisinės apsaugos įstatymas
	\item Darbo kodeksas
\end{enumerate}
\begin{table}[H]\footnotesize
	\centering
	\caption{Įvaizdis}
	{\begin{tabular}{|c|c|c|c|} \hline
			Vertinimo kriterijus & Mertinimo matas & Siekiama vertė & Kritinė vertė \\
			\hline
			Asmens duomenų teisinės apsaugos įstatymas & Pažeidimų skaičius & 0 & 0 \\
			\hline
			Darbo kodeksas & Viršytas darbo laikas (kartais per metus) & 0 & 0 \\
			\hline
	\end{tabular}}
\end{table}
\begin{center}
	\subsection{Grėsmės}
\end{center}
\begin{center}
	\subsection{Neišnaudotos galimybės}
\end{center}
\end{document}